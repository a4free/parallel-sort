\section{Threading Building Blocks}
\label{sec:500}

Intel\tr developed the C++ Template Library Threading Building Blocks to offer a parallelization approach where the focus of the programmer could stay away from the parallelism intrinsic issues, and stay on the algorithm and \textit{useful} code.

The library attempts to provide an abstraction layer for the parallelism, effectively providing an almost completely different paradigm than the OpenMP library.

The most basic parallel operator in TBB is the \texttt{parallel\_for} function, which is the equivalent of \texttt{\#pragma omp parallel for} directive in OpenMP. This function takes as an argument a \textit{Functor}, or a class instance that defines the computation to be done in parallel.

TBB deals with the thread allocation and initialization, and also manages the workload, by dividing the range of iteration in chuncks. Each thread will receive a chunck that can be iterated just like a regular C++ collection. The \textit{Functor} allocated on that thread can then iterate through the chunck and perform the desired function, with full transparency from the threading and work assigning logic.

Besides the particular implementation details, the Radix Sort implementation follows a similar structure to the OpenMP implementation. Since Radix Sort consists mostly on array operations and iterations, there was no need for any TBB more complex operators like \texttt{parallel\_reduce} or \texttt{parallel\_pipeline}. However, the \texttt{concurrent\_vector} class proved useful for some specific issues that required the usage of explicit locks in the OpenMP version.