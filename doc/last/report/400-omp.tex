\section{OpenMP Implementation}
\label{sec:400}

For the OpenMP implementation used, 3 steps are required in each iteration:

\begin{enumerate}
	\item Bucket Fill
	\begin{itemize}
		\item[-] For $P$ threads and $B$ buckets, each thread will handle $B/P$ buckets
		\item[-] Each thread iterates entire array, inserting only on its own buckets
	\end{itemize}

	\item Master thread computes offset for each bucket on the result array

	\item Each thread copies its buckets to global array
\end{enumerate}

Given that radix sort is a non-comparative sorting algorithm, and it consists mostly on memory operations, it can be easily proven that it is completely memory bound. Because of that, no good results should be expected from this naive shared-memory implementation. The main bottleneck here is the fact that each thread iterates the array independently. Since each thread is not synchronized, no locality advantages will happen, and each thread will iterate at their own pace.

An alternative approach would be to iterate the array in smaller chuncks, synchronizing each thread after every chunck, to benefit from memory locality. However, because of time issues, and since the main goal of this work was not to obtain good results, but to study the implementation approaches, no effort was made in that direction.

Since the implementation relies on bitwise operators for simplicity and efficiency when computing bucket indexes, some limitations were introduced as a side effect. Due to the distribution of buckets for each thread, when the number of buckets is lower than the number of threads, since no verification is done, the result will contain errors. Also, the digit size must be a power of 2, to allow the number of iterations to be evenly calculated for the 32 bit keys.