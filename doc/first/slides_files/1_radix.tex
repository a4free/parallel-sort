\section{Radix Sort}

\begin{frame}
	\frametitle{Sequential Radix Sort}

	\begin{itemize}\itemsep=10pt
		\item Non-comparative sorting algorithm
		\item Works on integer or string keys
		\item Sort is done one digit at a time
		\item For each digit, set is iterated once
		\begin{itemize}
			\item[-] Complexity $O(k \cdot N)$, where $k$ is the maximum number of digits
		\end{itemize}

		\item Two versions
			\begin{itemize}
				\item[-] \textbf{Least Significant Digit}: Iterative, easy to implement, hard to parallelize
				\item[-] \textbf{Most Significant Digit}: Recursive, trivial to parallelize, but no load balance
				\item[-] This project considers only the LSD version
			\end{itemize}
	\end{itemize}
\end{frame}

\begin{frame}
   \frametitle{Sequential Radix Sort - Example}

	Initial array:
	\begin{lstlisting}^^J
		arr = [170, 45, 75, 90, 802, 24, 2, 66]^^J
	\end{lstlisting}

	First iteration (sort by first digit)
	\begin{lstlisting}^^J
		0: [170, 90],^^J
		2: [802, 2],^^J
		4: [24],^^J
		5: [45, 75],^^J
		6: [66]^^J
	\end{lstlisting}

	Result after first iteration:
	\begin{lstlisting}^^J
		arr = [170, 90, 802, 2, 24, 45, 75, 66]^^J
	\end{lstlisting}
	
\end{frame}

\section{Sequential}

\begin{frame}
	\frametitle{Sequential Radix Sort - Implementation}

	\begin{itemize}\itemsep=10pt
		\item Bitwise operators, and base 2 digits are used, allowing better performance.\\

		\item Amount of bits per digit ($g$) influences total number of buckets ($B = 2^g$)
		
		\begin{description}\itemsep=8pt
			\item[small $g$] Less buckets, but more iterations
			\item[big $g$] More buckets and less iterations, but more fragmentation and memory overhead
		\end{description}

	\end{itemize}

\end{frame}
